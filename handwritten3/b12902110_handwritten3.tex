 \documentclass[12pt, a4paper]{article}
\usepackage{enumitem}
\usepackage[left=2cm, right=2cm, top=2cm, bottom=2cm]{geometry}
\usepackage{graphicx}
\usepackage{minted}
\usepackage{xeCJK}

\renewcommand\arraystretch{1.1}
\setCJKmainfont[AutoFakeBold=1.5]{新細明體}

\setminted{
  fontsize=\footnotesize,
  frame=single,
}

\title{
  \vspace{-1cm}
  Systems Programming (2024 Fall)\\
  Handwritten Assignment 3
}
\author{\Large B12902110 呂承諺}
\date{}

\begin{document}
  \maketitle
  \begin{enumerate}
    \item We still use \verb|fork| if we want to spawn a child process that
    eventually calls \verb|exec| and runs a different executable.
    Otherwise, the parent process itself would be replaced.

    Moreover, if we want the new process to have its own address space (e.g.,
    heap, data, bss) and file descriptors, we would use \verb|fork()|.

    \item If a \verb|SIGALRM| is sent to the program from outside before
    \verb|setjmp(env_alrm)| is called, then \verb|env_alrm| would be undefined.

    Also, it's better to use \verb|sigsetjmp()| and \verb|siglongjmp()|,
    otherwise \verb|SIGALRM| may remain blocked after \verb|longjmp()| from
    \verb|sig_alrm()| back to \verb|main()|.

    \item
    \begin{enumerate}[label=(\Alph*)]
      \item \verb|pthread_cond_wait(&q->not_full, &q->mutex)|
      \item \verb|pthread_cond_singal(&q->not_empty)|
      \item \verb|pthread_cond_wait(&q->not_empty, &q->mutex)|
      \item \verb|pthread_cond_singal(&q->not_full)|
    \end{enumerate}

    We need the checks otherwise a deadlock would happen.
    Both functions would wait for the other's signal before signaling.

    \item Just like the example in the textbook.
    \inputminted{c}{4_ipc.c}
  \end{enumerate}
\end{document}
