\documentclass[12pt, a4paper]{article}
\usepackage{enumitem}
\usepackage{float}
\usepackage[left=2cm, right=2cm, top=2cm, bottom=2cm]{geometry}
\usepackage{graphicx}
\usepackage{pgfplots}
\usepackage{xeCJK}

\pgfplotsset{compat=1.18}

\renewcommand\arraystretch{1.1}
\setCJKmainfont[AutoFakeBold=1.5]{新細明體}

\title{%
  \vspace{-1.5cm}
  Systems Programming (2024 Fall)\\
  Programming Assignment 4
}
\author{\Large B12902110 呂承諺}
\date{}

\begin{document}
  \maketitle
  \begin{enumerate}
    \item The following are 3 cases with different $n$'s and $t$'s.
    For every request, \verb|num_works| is set to be equal to $n$, that is, every row is processed by a thread.

    \pgfplotsset{
      xlabel=Number of threads,
      ylabel=Running time (s),
    }
    \begin{tikzpicture}
      \begin{axis}[
        title={$n=2, t=10000$},
        scale=0.9
      ]
        \addplot table [header=false] {n2_t10000_result.txt};
      \end{axis}
    \end{tikzpicture}
    \hfill
    \begin{tikzpicture}
      \begin{axis}[
        title={$n=100, t=1000$},
        scale=0.9
      ]
        \addplot table [header=false] {n100_t1000_result.txt};
      \end{axis}
    \end{tikzpicture}

    \begin{center}
    \begin{tikzpicture}
      \begin{axis}[
        title={$n=2000, t=10$},
        width=\linewidth,
        height=6cm
      ]
        \addplot table [header=false] {n2000_t10_result.txt};
      \end{axis}
    \end{tikzpicture}
    \end{center}


    The first two cases doesn't show a significant trend. We suspect the
    frontend thread is the bottleneck.
    There are too many requests for he frontend thread to handle, but there is
    only one frontend thread.

    From the last case, we can see that running time decreases significantly
    when the number of threads increases from 1 to around 10, but remains
    roughly the same when the number of threads ranges from 10 to 100.
    The speedup disappears after 10 threads because there is bottleneck
    introduced by context switching.

    \pagebreak
    \item Synchronization is handled by checking the following
    three conditions:
    \begin{itemize}
      \item The frontend queue is empty.
      \item The worker queue is empty.
      \item There are no threads currently performing a job.
    \end{itemize}
    If all three conditions are fulfilled, then all threads are
    synchronized.

    The last condition is maintained by variable \verb|running_count|,
    representing the number of threads that has popped a request or work
    from the queue but is still processing it. This variable is
    protected by a mutex (\verb|running_mutex|).

    When a thread has finished its job, it signals the condition
    variable \verb|done|, which is waited by \verb|tpool_synchronize()|
    in a while loop the check for the three conditions above.
    This way, \verb|tpool_synchronize()| will not busy wait.
  \end{enumerate}
\end{document}
