\documentclass[12pt, a4paper]{article}
\usepackage{enumitem}
\usepackage[left=2cm, right=2cm, top=2cm, bottom=2cm]{geometry}
\usepackage{graphicx}
\usepackage{xeCJK}

\renewcommand\arraystretch{1.1}
\setCJKmainfont[AutoFakeBold=1.5]{新細明體}

\title{
  \vspace{-1cm}
  Systems Programming (2024 Fall)\\
  Programming Assignment 1
}
\author{\Large B12902110 呂承諺}
\date{}

\begin{document}
  \maketitle
  \begin{enumerate}
    \item
    \begin{enumerate}
      \item Busy waiting means a process is repeatedly testing if a condition
      has become true. In this assignment, that is testing if data is available
      to be read.

      \item In this assignment, we can avoid busy waiting by using
      \verb|select()| or \verb|poll()| with a properly configured timeout.
      This way, the server only continues execution when there is data is ready
      to be read.

      \item It is still possible to have busy waiting even with \verb|select()|
      or \verb|poll()|, if the timeout is set to 0.
    \end{enumerate}

    \item
    \begin{enumerate}
      \item Starvation refers to an I/O request not getting data, resulting in
      an indefinite wait.
      \item In this assignment, it is possible for a request to encounter
      starvation if our code handles a request using infinite loops.
      To avoid this, we use a state-based approach, keeping track of every
      request's state and changing its state accordingly after each command.
      This way, we only have to handle commands when input data is available,
      thus preventing indefinite waits.
    \end{enumerate}

    \item Since a process could never test for its own lock via \verb|fcntl()|,
    we need to keep track of which seats are reserved or paid by which requests
    handled by this server process. This can be done by creating an array of
    seat statuses for each shift, maintaining the request's ID that has reserved
    or paid for the shift.

    \item We use advisory record locking provided by \verb|fcntl()|.
    Before writing a byte range (a seat is essentially a byte) in a shift's record
    file, we first need to check is another process is holding a lock to that
    byte range. If that is true, we have to ensure that our process will not write to
    that byte range.

  \end{enumerate}
\end{document}
